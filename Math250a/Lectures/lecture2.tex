% !TEX root = lectures.tex
\section{Lecture 2}

\subsection{Unique Factorization Domains}
We wish to show today that all principal ideal domains are \textbf{Unique Factorization Domains}.
For this lecture, we will assume $R$ denotes a principal ideal domain. We wish to show that
for $r \in R$, $r$ admits a unique factorization in terms of irreducible elements. 
\begin{definition}
    An irreducible element $i \in R$ is an element that has no divisors except $\pm$ itself and $\pm 1$ and units.
\end{definition}
\begin{definition}
    An element $p \in R$ is prime if $rs \in (p) \implies r \in (p)$ or $s \in (p)$.
\end{definition}
\begin{theorem}
    Every prime element is irreducible.
    \begin{proof}
        Suppose $p$ is prime and you could factor it as $p = ab$. By primality, $a$ or $b$ is divisible by $p$, without loss
        of generality this is $a$. Then $a = kp$ for some $k$, so $p = kbp$ or $(kb - 1)p = 0$. Thus $kb - 1 = 0$ and $kb = 1$, so $b$ and $k$ must be units.
        Thus, $p$ is irreducible.
    \end{proof}
\end{theorem}
The algorithm for
creating this factorization is simple, if you have an irreducible element, just leave it. Otherwise it must be reducible; take that
factor out and continue.
Thus, to prove the claim, it's sufficient to show that this algorithm terminates. In other words, any chain of ideals
has a largest element:
\[ (r_1) \subset (r_2) \subset (r_3) \subset \dots \subset (r) \]
If we have such a chain, note that it's finite by the following idea. Consider the union $\bigcup_i (r_i)$.
Since this is an ideal and this is a PID, $\bigcup_i (r_i) = (r)$ for some $r \in R$. Furthermore, $r$
must exist in one such ideal; that ideal must include $(r)$, so it must be exactly $(r)$. 
This property of all such chains of ideals being finite is called the \textit{Noetherian} property.
These kind of \textit{Noetherian} rings are typically those that are finitely generated.
\begin{theorem}
    Every irreducible element of a PID are prime.
    \begin{proof}
        Suppose $rs \in (p)$ for some $r,s \in R$. Suppose $p\in R$ is irreducible. Suppose $r \notin (p)$. But this means that $(r, p) \supsetneq (p)$.
        Since $R$ is a PID, this means $(r, p) = (a)$ for some $a \in R$. Thus, $p = au$ for some $u \in R$
        Thus, $a$ is a unit, so $(a) = (1) = (r, p)$. That means for some $x, y$, we can write $1 = rx + py$. Multiplying by $s$,
        then $s = rxs + pys = (rs)x + pys$, so $s \in (p)$. Thus $p$ is prime.
    \end{proof}
\end{theorem}
Now to proceed with the proof of factorization. By this algorithm,
we know we can write $0 \neq r = \prod_{i = 1}^m p_i^{a_i}$ as a product of primes (which are the same as irreducibles).
Suppose there was another factorization $r = \prod_{i = 1}^n q_i^{b_i}$. We claim that $\{p_i\}$ and $\{q_i\}$ (and associated exponents)
are just the up to permutation and units. The proof is induction on $\sum_i a_i$: just take one of the primes on the left; it must divide one of the factors on the right
by the definition of prime. Thus, divide on both sides and you reduce the $a_i$s by $1$ (perhaps you get some units as left-overs, we can ignore these).
\subsection{Classification of Finitely-Generated Modules (Cont'd)}
Recall the theorem we attempted to show last time.
\begin{theorem}
    Suppose $M$ is a finitely-generated module over a PID. then $M \cong \bigoplus_i M_i$,
    where each $M_i$ is cyclic (generated by one element).
\end{theorem}
Multiplication by an element of a ring becomes a homomorphism on modules; in general this is a representation: which turns
group elements into transformations. Recall we started the proof with the following construction.
Take the torsion submodule
\[ M_{\text{tors}} = \{m \in M \mid \exists r \neq 0 \in R, rm = 0 \} \]
The claim is that $\qty(M/M_{\text{tors}})_{\text{tors}} = \{0\}$, i.e. $M_{\text{tors}}$ is torsion-free.
Consider $\overline{m} \in M/M_{\text{tors}}$ such that $r \overline{m} = 0$ for some $r \neq 0$.
This means that $rm \in M_{\text{tors}}$, so there exists $s \in R$ which is nonzero such that $srm = 0$.
Since $m \in M_{\text{tors}}$, we're done. Consider the canonical homomorphism $M \to M/M_{\text{tors}}$. Why don't we just pick
one representative from each coset? Usually this doesn't create a submodule, but it does here because the module is free.
\begin{theorem}
Any torsion-free finitely-generated
module over a PID $R$ is free (which means $\cong R^{\oplus n} = R^n$).
\end{theorem}
We first need the following lemma.
\begin{lemma}
    If $M \subset R^n$ is a submodule of the free module of rank $n$, then $M$ is free of rank $\leq n$.
\end{lemma}
\begin{definition}
    If $p \in R$ is prime, then $R/(p)$ is a field. Thus for any free $R$-module $M$, $M/pM$ is a module over $R/(p)$ (in other words,
    a vector space). The rank of $M$ is the rank of this vector space. Rank is well-defined for free modules. Equivalently,
    we can say that the rank is the maximal set of linearly independent elements that generate the module.
\end{definition}
Clearly $\rank R^n = \dim_{R/(p)} R^n/pR^n = (R/(p))^n$. Now let's prove our lemma by induction on $n$.
If $n = 1$, then we have $M \subset R$. This means it's a principal ideal $(a) \subset R$ (as rings), but as $R$-modules,
$(a)_{\text{module}} = aR \cong R^1$. Then for the inductive step, we know
We know that $R^{n - 1} \subset R^n$, so we have the exact sequence
\[ 0 \to R^{n - 1} \to R^n \to_{\phi} R \to 0 \]
we can rewrite this exact sequence for some $a \in R$:
\[ 0 \to M \cap R^{n - 1} \to M \to (a) \to 0 \]
Call $R^n = \bigoplus_{i = 1}^n R f_i$. Then we can decompose $m \in M$ as
\[ m = \sum_{i = 1}^n r_i f_i = \sum_{i = 1}^{n - 1} r_i f_i + r_n f_n \]
This means $\phi(m) = r_n$. This means $M = M \cap R^{n - 1} \oplus a R f_n$. The first one is a subset of $R^{n - 1}$,
so it is a module of rank at most $n - 1$ (by induction, free) and the second one is just $R$ (so, free). Thus we get rank $n$.
\begin{lemma}
    If $R$ is a PID and $M$ is finitely generated over $R$ and $M' \subset M$, then $M'$ is finitely generated.
    \begin{proof}
        There exists a surjective homomorphism $\phi: R^n \to M$ for some $n$, by the definition of direct sum.
        Call $M' \subset M$ and call $F = \phi^{-1}(M')$. By lemma, $F$ is a free module of rank at most $n$
        and we have a surjective homomorphism from it to $M'$. Thus,
        it is generated by at most $n$ elements.
    \end{proof}
\end{lemma}
Now we can prove the theorem. Suppose $M$ is torsion-free that is finitely generated. Let's take a maximal set of linearly independent elements from $M$
(note that this is always finite; if we have an increasing chain of inclusions, the module is finitely generated
so there exists a finite set that contains every submodule). Call this set
\[ f_1, \dots, f_n \text{ where if }  \sum_n r_n f_n = 0, r_n \in R \implies \text{all } r_n = 0 \]
Now $M/(f_1, \dots, f_n)$ has torsion, because if $g \in M, g \notin (f_1, \dots, f_n)$ then there exists $r_i$'s and $r$ such that
$\sum_i r_i f_i + rg = 0$ where not all the coefficients are $0$ (and $r$ cannot be either). So $r \cdot \overline{g} = 0$. Thus, $M/(f_1, \dots, f_n)$
has all elements torsional.

Now consider all such $g$ which are generators. This shows that if we take their $r$'s and multiply them together to make $s \neq 0$,
we can annihilate these generators and thus $sM \subset \sum_i R_i f_i \cong R^n$. But $M \cong sM$. So $M$ is free.
We claim this means that \[ M \cong M_{\text{tors}} \oplus M/M_{\text{tors}} \]
Clearly these are free modules--we just need to show that the canonical homomorphism is a splitting map, meaning
it truly creates a direct sum.
\begin{proof}
    Suppose $M/M_{\text{tors}} = \bigoplus_{i = 1}^n R \overline{f}_i$ for some $f_i \in M$. Consider $\bigoplus R f_i \subset M$,
    where $f_i$ are some representatives of the barred versions.
    By our theorem, $\bigoplus R f_i$ is free and $\bigoplus R f_i \cap M_{\text{tors}} = 0$. Also, we can write $m \in M$ as $m' + m''$ with $m' \in M/M_{\text{tors}}$ and $m'' \in M_{\text{tors}}$,
    by the definition of quotient. Thus, the direct sum is indeed valid.
\end{proof}
\begin{theorem}
    If $M$ has torsion and finitely generated, then $M$ naturally splits as $M \cong \bigoplus_{\text{primes }p} M(p)$
    where $M(p) = \{m \in M \mid p^k m = 0 \text{ for some $k \geq 0$}\}$.
    \begin{proof}
        There exists a nonzero element $r \neq 0 \in R$ such that $rM = 0$. In fact $M = \bigoplus_{p \mid r} M(p)$.
    \end{proof}
\end{theorem}